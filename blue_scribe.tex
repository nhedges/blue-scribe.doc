\documentclass[11pt]{LaTeX-Classes/math-hw}
\usepackage{float}
\usepackage{dirtytalk}
\title{Final Project: Computer Controlled Laser Engraver}
\firstname{Nicholas Hedges and  Wyatt Wayman}
\lastname{}
\coursename{ECE 3710 Fall 2020}
\professorname{Dr. Phillips -- Nathan Burk}
\date{15 December 2020}

\begin{document}
\maketitle

\section*{Introduction}
For our final project in ECE 3710, we are using our microcontroller and a PC to implement a 2D laser engraver.
To do this, we implemented code to scan an image file and break the image down into motor and laser instructions.
Following this, the PC would send these instructions serially to the microcontroller via USB. The microcontroller
uses the UART to receive and parse the instructions so they can be turned into motor and laser commands. The result
is a fully functional laser engraver.

\section*{Scope}
This document discusses the design overview, design details, and testing methods for our laser engraver.

\section*{Design Overview}
\subsection{Requirements}
Our laser engraver shall meet the following functionality requirements:
\begin{itemize}
	\item The laser engraver shall be able to etch at least a 100 mm by 100 mm square.
	\item The laser and x-y actuators shall be mounted within their own secure housing.
	\item The laser shall be securely attached to the x-y actuators.
	\item The laser brightness shall be changed for different instruction purposes.
	\item The laser location shall be able to return a home position with a single instruction.
	\item The microcontroller shall be able to communicate serially with the PC via USB.
	\item The program for turning images into microcontroller instructions shall ????*****
	\item The microcontroller shall be able to receive instructions manually through a terminal on the PC.
	\item The microcontroller shall be able to receive instructions automatically from the image breakdown program.
	\item The microcontroller shall be able to send a response flag via the UART back to the PC upon instruction completion.
	\item The microcontroller shall be able to send an error flag via the UART back to the PC upon invalid instruction.
	
\end{itemize}
\subsection{Dependencies}
\subsection{Theory of Operation}
\subsection{Design Alternatives}


\section*{Design Details}


\section*{Testing}
\subsection{Size Test}
\subsection{Motor Functionality Test}
\subsection{Homing Test}
\subsection{Laser PWM Test}
\subsection{USB UART Communication Test}
\subsection{Image Instruction Breakdown Test}
\subsection{Manual Instruction Input Test}
\subsection{Computer Generate Input Test}

\section*{Conclusion}


\section*{Appendices}
\subsection{Figures}
 \begin{figure}[H]
   \begin{center}
     \includegraphics[width=0.6\textwidth]{blocks}
     \caption{Top level design for the flow of states of our laser engraver}
     \label{fig:blockdiagram}
   \end{center}
 \end{figure}

\begin{figure}[H]
	   \begin{center}
	     \includegraphics[width=0.6\textwidth]{blocks}
	     \caption{Top level design for the flow of interrupts and timers.}
	     \label{fig:mcwiringdiagram}
	   \end{center}
	 \end{figure}
\end{document}
